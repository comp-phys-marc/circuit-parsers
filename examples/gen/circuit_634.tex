
\documentclass{article}
\usepackage{qcircuit}
\begin{document}
\Qcircuit @C=1em @R=.7em {
 & \ctrl{1} & \gate{S} & \qw \\ 
 & \targ & \gate{Y} & \gate{S^\dagger} \\ 
}
\end{document}
        